\chapter{绪论}

\section{研究背景}

有人曾经将20 世纪称之为新闻媒体“裂变”的世纪。因为在20 世纪中,多少个
风风雨雨从未有过变化的,仅只有报纸、杂志等的平面媒体的传播体系,却在这个世
纪中发生了翻天覆地的变化。在人们的生活中,不仅可以见到广播、电视这样的新媒
体,而且还出现了可以综合应用各种媒体的新媒体——网络。在网络出现后,它就像
病毒一样,疯狂的侵入了我们的生活。据统计,自从网络在1994 年以来经历了商业
化的运作后,它便以人们意想不到的速度发展着。比如在达到5000 万的用户上所花
费的时间,报纸用了近一个世纪,无线广播用了38 年,而如今早已渗入到每个家庭
中的电视媒介也同样历经了13 年的发展时间才达到这个规模,而看看网络,它仅用
了4 年时间,就已经站上了这样的一个高度,成为了新兴的“第四类媒介”。同时,它
的用户增长正以其他媒介望尘莫及的速度持续上升着。正是由于网络具有这样庞大的
用户基数,以及具有的其他特性便推动了网络广告的诞生和发展。
